%%%%%%%%%%%%%%%%%%%%%%%%%%%%%%%%%%%%%%%%%
%%%%%%%%%% Content starts here %%%%%%%%%%
%%%%%%%%%%%%%%%%%%%%%%%%%%%%%%%%%%%%%%%%%

\begin{frame}[fragile]{Задача 1}
\begin{tiny}
Драмскиот театар во Скопје се одлучил да воведе систем во кој секој посетител
на драмска претстава добива троцифрен број и според овој број се одредува на кој
од трите влеза влегува посетителот и за каков посетител се работи (ВИП или
регуларен). Видот на посетителот се одредува во зависност од првата цифра на
троцифрениот број, при што ако цифрата е: 
\begin{itemize}
  \item 1, се работи за ВИП посетител
  \item 2 - 9, се работи за регуларен посетител  
\end{itemize}

Влезот на кој треба да влезе посетителот се одредува според остатокот при делење
на бројот со 3. Ако остатокот е: 
\begin{itemize}
  \item 0, се влегува на првиот влез
  \item 1, се влегува на вториот влез
  \item 2, се влегува на третиот влез.  
\end{itemize}
Ваша задача е да напишете програма во која од СВ се чита број и според овој број
на стандарден излез се печати за каков посетител се работи и на кој влез треба
да влезе. Ако бројот кој се внесува не е троцифрен се печати порака за погрешно внесен број.


\emph{Пример}:
\begin{verbatim}
	Влез:				Излез:
    135				    VIP posetitel na vlez 1
\end{verbatim}
\end{tiny}

\end{frame}

\begin{frame}[fragile]{Задача 1}{Решение}
\lstinputlisting{src/kol/p1.c}
\end{frame}

\begin{frame}[fragile]{Задача 2}
Да се напише програма во која се читаат знаци од стандарден влез сѐ додека не се
прочита знакот ‘.’. Програмата треба да ги отпечати на стандарден излез знакот
со најмногу последователни појавувања како и колку е вкупниот број на вакви
последователни појавувања. Ако повеќе знаци имаат ист број на последователни
појавувања, да се отпечати првиот ваков знак.

\emph{Пример:}
\begin{verbatim}
	Влез:				
		ova-Treba da Elesnaaaaa_zAaaadddddacca.
	Излез: 
		a 5
\end{verbatim}
\end{frame}
\begin{frame}[fragile,shrink=.95]{Задача 2}{Решение}
\lstinputlisting{src/kol/p2.c}
\end{frame}


\begin{frame}[fragile]{Задача 3}
\begin{tiny}
Во националната лига во фудбал учествуваат \texttt{n} (\texttt{n > 2}) екипи.
Секоја екипа во лигата одигрува по 2 натпревари (еден дома и еден во гости) со секоја друга
екипа. Притоа, победничкиот тим на еден натпревар освојува 3 поени, додека
поразениот не освојува поени. Во случај на нерешен резултат, на двата тима им се
доделува по 1 поен.
Да се напише програма која од СВ прво ќе прочита природен број \texttt{n} - бројот на
екипи, а потоа и информации за победите, нерешените резултати и поразите на 
\textbf{n}-те екипи по завршувањето на лигата. Програмата треба на СИ да го
отпечати максималниот број на поени освоени од страна на една екипа, како и
редниот број на екипата (според редоследот на внесување: првата екипа за која се
внесуваат податоци има реден број 1, втората има реден број 2, итн.).
Дополнително, ако максималниот број на освоени поени е над 75\% од максималниот
број на поени што може да ги освои секоја екипа - учесник во лигата, програмата
треба да отпечати \texttt{``Odlicna sezona''}, во спротивно треба да отпечати
\texttt{``Losa sezona''}.
\emph{Пример:}
\begin{verbatim}
	Влез:				Излез:
	6               Maksimalen broj na osvoeni poeni: 23
	6   3   1	    Reden broj na ekipata: 3
	2   2   6	    Odlicna sezona
	7   2   1
	0   0   10
	4   3   3
	5   2   3
\end{verbatim}
\end{tiny}
\end{frame}

\begin{frame}[fragile]{Задача 3}{Решение}
\lstinputlisting{src/kol/p3.c}
\end{frame}

\begin{frame}[fragile]{Задача 4}
Да се напише програма која од СВ ќе прочита еден природен број \texttt{n}, и на
СИ ќе го испечати најмалиот „среќен“ број кој што e поголем од \texttt{n}. За еден
природен број велиме дека е „среќен“ ако тој содржи само „среќни“ цифри (цифрите
7 и 9).
Пример:
\begin{verbatim}
	Влез:			Излез:
	789				797
\end{verbatim}

\end{frame}

\begin{frame}[fragile]{Задача 4}{Решение}
\lstinputlisting{src/kol/p4.c}
\end{frame}

\begin{frame}[fragile]{Задача 5}
Да се напише програма која ќе ги проверува броевите од 1 до 100. Притоа, за
секој број кој што е делив со 3 наместо бројот ќе отпечати \texttt{``Tip''}, а за секој
број кој што е делив со 5 наместо бројот ќе отпечати \texttt{``Top''}. Наместо
броевите кои се деливи и со 3 и со 5 ќе отпечати „TipTop“.
\end{frame}

\begin{frame}[fragile]{Задача 5}{Решение}
\lstinputlisting{src/kol/p5.c}
\end{frame}
