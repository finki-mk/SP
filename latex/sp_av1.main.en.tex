
\section{Introduction to Programming}

\begin{frame}{Programming (1)}
\begin{itemize}
\item Executable programs are containing only \textbf{zeros and ones}, because
that is the only language that computer understands
\item Programmers write the programs in human readable
languages, called \textbf{programming languages}
\item Program written in programming language is called a \textbf{source code}
\end{itemize}
\end{frame}

\begin{frame}{Programming (2)}
\begin{itemize}
  \item Usually for writing source code programmers use \textbf{development
  environments}
  \item The code is written in text editor
  \item Then the code is (\textbf{compiled})
  \item At the end we have \textbf{executable} i.e. program written in machine
  language (the language of the computer)
\end{itemize}
\end{frame}

% Define block styles
\tikzstyle{block} = [rectangle, draw, fill=blue!20,
text width=5em, text centered, rounded corners, minimum height=4em]
\tikzstyle{line} = [draw, -latex']

\begin{frame}[fragile,shrink=20]{Compiling and executing process }

\begin{center}
Phase 1 - compiling the source code
\begin{tikzpicture}[node distance = 3cm, auto]
    % Place nodes
    \node [block] (compiler) {Compiler};
    \node [block, below of=compiler, node distance=2cm] (source) {Source code}; 
    \node [block, right of=compiler] (computer) {Computer};
    \node [block, right of=computer] (executable) {Executable (in memory)};
    % Draw edges
    \path [line] (compiler) -> node {} (computer);
    \path [line] (source) -| node {} (computer);
    \path [line] (computer) -> node {} (executable);
\end{tikzpicture}
\end{center}
\begin{center}
Phase 2 - execution
\begin{tikzpicture}[node distance = 3cm, auto]
    % Place nodes
    \node [block] (program) {Executable (in memory)};
    \node [block, below of=program] (data) {Input data};
    \node [block, right of=program] (computer) {Computer};
    \node [block, text width=3cm, right of=computer] (executable) {Execution
    result};
    % Draw edges
    \path [line] (program) -> node {} (computer);
    \path [line] (data) -| node {} (computer);
    \path [line] (computer) -> node {} (executable);
\end{tikzpicture}
\end{center}
\end{frame}


\begin{frame}{Introduction to C programming language}
\begin{itemize}
\item Developed in Bell laboratories in period from 1969 to 1973 by Dennis
Ritchie
\item One of the most used general purpose programming languages of all time
\item Has big influence in origins and development of many other programming
languages
    \begin{itemize}
    \item C++
    \item Objective C
    \item PHP
    \item Java
    \end{itemize}
\end{itemize}
\end{frame}

\begin{frame}[fragile]{C Syntax}

  \begin{block}{Allowed characters in C:}
  \begin{verbatim}
    a-z, A-Z, 0-9 и ~!@#$%^&*()-+={}[]:;'"<>?/._  
  \end{verbatim}
  \end{block}

    \begin{alertblock}{Attention!}
    Compiler is case sensitive (makes difference between uppercase and lowercase)!
    \end{alertblock}

    All characters in C are forming words that can be:
    \begin{enumerate}
    \item Keywords
    \item Numerical and symbolic constants
    \item Identifiers
    \item Strings (char arrays)
    \item Operators
    \end{enumerate}

\end{frame}

\begin{frame}{C Syntax}

    \begin{block}{Set of keywords (32)}

    \begin{tabular}{c c c c}
        \texttt{auto} & \texttt{double} & \texttt{int} & \texttt{struct} \\
        \texttt{break} & \texttt{else} & \texttt{long} & \texttt{switch} \\
        \texttt{case} & \texttt{enum} & \texttt{register} & \texttt{typedef} \\
        \texttt{char} & \texttt{extern} & \texttt{return} & \texttt{union} \\
        \texttt{const} & \texttt{float} & \texttt{short} & \texttt{unsigned} \\
        \texttt{continue} & \texttt{for} & \texttt{signed} & \texttt{void} \\
        \texttt{default} & \texttt{goto} & \texttt{sizeof} & \texttt{volatile} \\
        \texttt{go} & \texttt{if} & \texttt{static} & \texttt{while}
    \end{tabular}
    \end{block}
\end{frame}

\begin{frame}[fragile]{Simple C program structure}

    All the source code written in C is organized in \textbf{functions}
    \linebreak
    
    \begin{columns}[t]
        \column{.5\textwidth}
            \begin{block}{Program in C}
                \begin{lstlisting}
                int main() {
                    variable_declaration;
                    expressions;                    
                }
                \end{lstlisting}
            \end{block}
        \column{.5\textwidth}
            \begin{block}{Program in Pascal}
                \begin{lstlisting}
                Program program_name;
                var variable_declaration;
                begin
                    expressions;;
                end.
                \end{lstlisting}
            \end{block}
    \end{columns}

\end{frame}

\begin{frame}[shrink=10]{C Functions}
    \hfill
    \begin{block}{\texttt{main}}
         C main function
    \end{block}
    
    \begin{block}{\texttt{()}}
         In parentheses we put the input arguments
    \end{block}
    
    \begin{block}{\texttt{int}}
         Data type of the result it's before the function name
    \end{block}
    
    \begin{block}{\texttt{\{\}}}
         The function body starts with \{, and ends with \}
    \end{block}

    \begin{block}{\texttt{;}}
    All declarations and expressions are separated with \texttt{;}
    \end{block}

\end{frame}

\begin{frame}[fragile]{Comments usage}
    \begin{itemize}
        \item Comments are used for extra explanation or documenting the source code
        \item C supports two types of comments
        \begin{enumerate}
            \item one line comments
            \item multiple line comments
        \end{enumerate} 
        
    \end{itemize}
        
    \begin{columns}[t]
        \column{.5\textwidth}
            \begin{block}{1. One line comments}
                \begin{verbatim}
                // one line comment ;)
                \end{verbatim}
            \end{block}
        \column{.5\textwidth}
            \begin{block}{2. Multiple line comment}
                \begin{verbatim}
                /* Comment
                   in multiple lines */
                \end{verbatim}
            \end{block}
    \end{columns}       

\end{frame}

\begin{frame}[fragile]{Examples}
    \begin{exampleblock}{Example 1}
        \begin{lstlisting}
            #include <stdio.h>
            // main function
            int main() {
                /* function for print on standard output (screen) */
                printf("Welcome to FINKI!\n");
                return 0;
            }
        \end{lstlisting}
    \end{exampleblock}
\end{frame}

\begin{frame}{C program structure (extended)}
    \begin{columns}
        \column{.2\textwidth}
        \textbf{INCLUDE} section
        \column{.8\textwidth} contains \texttt{\#include} expressions for
        including external libraries, for using externally declared functions
    \end{columns}
    \hfill
    \linebreak
    \begin{columns}
        \column{.2\textwidth}
        \textbf{DEFINE} section
        \column{.8\textwidth} contains \texttt{\#define} expressions for constants and
        data type declaration
    \end{columns}   
    \hfill
    \linebreak
    \begin{columns}
        \column{.2\textwidth}
        ...
        \column{.8\textwidth} defining global variables and functions
    \end{columns}
    \hfill
    \linebreak
    \begin{columns}
        \column{.2\textwidth}
        \texttt{int main()}
        \column{.8\textwidth} main function
    \end{columns}
\end{frame}

\begin{frame}{Preprocessor}
\begin{itemize}
    \item In C compiling is done by:
    \begin{itemize}
        \item preprocessor
        \item compiler
    \end{itemize}
    \item The preprocessor is used with directives
    \begin{itemize}
        \item Each directive starts with \texttt{\#}
    \end{itemize}
\end{itemize}
\end{frame}

\begin{frame}{Header files}
\begin{itemize}
    \item One usage of preprocessor directive is including the ``header files''
    \begin{itemize}
        \item Used for declaration of functions and variables in some predefined
        library
        \item In order to use the external functions and variables users include the
        ``header file''
    \end{itemize}
    \item The inclusion is done with the preprocessor directive 
    \texttt{\#include}
    \begin{itemize}
        \item Statement \texttt{\#include} causes inclusion a copy of the given file
        in the place where the statement is written
    \end{itemize}
\end{itemize}
\end{frame}

\begin{frame}[fragile]{\texttt{include} variations}
\begin{itemize}
    \item There are two types of this directive:
    \begin{itemize}
        \item file that is included can be written in double quotes (""),
        \item or in angle brackets (<>)
    \end{itemize}
    \begin{exampleblock}{Example}
        \begin{verbatim}
        #include <filename.h>
        #include "filename.h"
        \end{verbatim}
    \end{exampleblock}
    \item The difference is in the location in which the preprocessor
    searches the file that should be included
    \begin{itemize}
        \item In angle brackets (files from the standard library are used)
        \item With double quotes (the preprocessor first tries to find
        the file in the same directory with the C file that should be
        compiled)
    \end{itemize}
\end{itemize}

\end{frame}

\begin{frame}{Variables}
\begin{itemize}
\item Variables are symbolic names for places in memory where the
computer keeps values
\item Before usage all variables must be \emph{declared}
\item With each new assignment of value, the old value is discarded
\end{itemize}

Types of variable declaration:
\linebreak
\begin{columns}
        \column{.32\textwidth} \fbox{Variable type}
        \column{.32\textwidth} \fbox{Variable name}
        \column{.05\textwidth} \fbox{ = }
        \column{.26\textwidth} \fbox{Initial value}
        \column{.05\textwidth} \fbox{ ; }
\end{columns}   

\end{frame}

\begin{frame}{Variable types}
\Large{Variable types in C}
\linebreak
\linebreak
\begin{tabular}{c|c|c}
\textbf{Numbers} & \textbf{Letters} & \textbf{Decimal}\\
\hline
\texttt{int} & \texttt{char} & \texttt{float} \\
\hline
\texttt{short} & & \texttt{double} \\
\hline
\texttt{long} & &
\end{tabular}
\end{frame}

\begin{frame}{Defining variable names}
\begin{itemize}
\item In naming variables allowed are:
\begin{itemize}
\item lowercase letters from a to z;
\item uppercase letters from A to Z;
\item digits from 0 to 9 (can't start with digit);
\item underscore \_ which is recognized as a letter (not recommended to start
with \_);
\end{itemize}
\end{itemize}
\begin{alertblock}{Be carefull!}
\begin{itemize}
\item mostly names length is up to 32 characters
\item С is case sensitive!
\end{itemize}
\end{alertblock}
\end{frame}

\begin{frame}[fragile]{Examples}
    \begin{exampleblock}{Example 2}
        \begin{lstlisting}
            #include <stdio.h>

            int main() {
                int a, b, c;
                a = 5;
                b = 10;
                c = a + b;
                return 0;
            }
        \end{lstlisting}
    \end{exampleblock}
\end{frame}

\begin{frame}{Constants}
\begin{itemize}
\item Constants are used for values that are unchanged during the program
execution
\item Each constant is from some data type
\item C has several constant types:
\begin{itemize}
\item decimal: \texttt{1, -23, 15}
\item octal: \texttt{015, 035, 0205}
\item hexadecimal: \texttt{0x25, 0xA4C}
\item real: \texttt{3.5F, -2.845F, 1.34e-9}
\item literals: \texttt{'a', '\_', 'e'}
\item strings: \texttt{" ", "Koncepti za razvoj na softver"}
\end{itemize}
\end{itemize}
\end{frame}

\begin{frame}{Determining constant type}
\begin{itemize}
\item Determining the type of varialbes is simple (can be seen from the
declaration)
\item Constants aren't declared, so their type is determined by the
way they are written:
\begin{itemize}
\item Numbers that contain "." or "е" are
{\color{blue}\texttt{double}}:
\texttt{3.5, 1е-7, -1.29е15}
\item Instead of using \texttt{double}, to use {\color{blue}{\texttt{float}}}
constants "F" is added at the end: \texttt{3.5F, 1e-7F}
\item For {\color{blue}\texttt{long double}} constants "L" is added:
\texttt{1.29е15L, 1e-7L}
\item Numbers without ".", "е" or "F" are {\color{blue}\texttt{int}}:
\texttt{1000, -35}
\item For {\color{blue}\texttt{long int}} constants "L" is added:
\texttt{9000000L}
\end{itemize}
\end{itemize}
\end{frame}

\begin{frame}[fragile]{Named constants (1)}
\Large{Named constants are created by using the keyworkd \texttt{const}}
\begin{exampleblock}{Example 3}
        \begin{lstlisting}
            #include <stdio.h>

            int main() {
                const long double pi = 3.141592653590L;
                const int days_in_week = 7;
                const sunday = 0; /* by default int */
                days_of_week = 1; /* error */
            }
        \end{lstlisting}
    \end{exampleblock}
\end{frame}

\begin{frame}[fragile]{Named constants (2)}
Named constants can be created with the preprocessor, and usually
uppercase letters are used
    \begin{exampleblock}{Example 3}
        \begin{lstlisting}
            #include <stdio.h>
            #define PI 3.141592653590L
            #define DAYS_IN_WEEK 7
            #define SUNDAY 7        
            int main() {
                long double pi = PI;
                int day = SUNDAY;
            }
        \end{lstlisting}
    \end{exampleblock}
\end{frame}

\begin{frame}{Operators}
\begin{itemize}
\item Operators are used in builing expressions, and operations are
evaluated from left to right by applying the operator priority rules
\item There are three types of operators
\begin{itemize}
\item Arithmetic operators
\item Relational operators
\item Logical operators
\end{itemize}
\end{itemize}
\end{frame}

\begin{frame}{Arithmetic operators}
Usen on numbers (integers or reals)
\linebreak
\begin{center}
\begin{tabular}{c|c}
\textbf{Operator} & \textbf{Operation}\\
\hline
\texttt{+} & Addition \\
\texttt{-} & Substruction \\
\texttt{*} & Multiplication \\
\texttt{/} & Division \\
\texttt{\%} & Division by modulo
\end{tabular}
\end{center}
\end{frame}

\begin{frame}{Relational operators}
Applyed on any comparable data types, and the result is 0 (false) or 1 (true).
\begin{center}
\begin{tabular}{c|c}
\textbf{Operator} & \textbf{Meaning}\\
\hline
\texttt{<} & Less than \\
\texttt{<=} & Less than or equal \\
\texttt{>} & Greater than \\
\texttt{>=} & Greater than or equal \\
\texttt{==} & equal \\
\texttt{!=} & not equal
\end{tabular}
\end{center}
\end{frame}

\begin{frame}{Logical operators}
Used mostly in combination with relational operators in forming complicated
logical expressions, which again results in 0 or 1
\linebreak
\begin{center}
\begin{tabular}{c|c}
\textbf{Operator} & \textbf{Operation}\\
\hline
\texttt{\&\&} & Logical AND \\
\texttt{||} & Logical OR \\
\texttt{!} & Negation
\end{tabular}
\end{center}
\end{frame}

\begin{frame}[fragile]{Additional operators}
\begin{itemize}
\item Assignment operator =
\item Increment and decrement operators
\begin{verbatim}
++, --
\end{verbatim}
\item Unary usage of operators + and –
  \begin{verbatim}
    X = + Y;
    X = - Y;
  \end{verbatim}
\item Double operators
\begin{itemize}
\item Combination of assignment operator and other operator
\begin{verbatim}
(+=, -=, *=, /=, %=)
\end{verbatim}
\end{itemize}
\end{itemize}
\end{frame}

\begin{frame}[fragile]{Assignment operator =}
\begin{itemize}
\item All expressions have values, even those containing =
\item Value of such expression is the value of the expression of the
right side
\item Because of that, this assignment is possible:
\end{itemize}
\begin{verbatim}
    x = (y = 10) * (z = 5);
    x = y = z = 20;
\end{verbatim}
\end{frame} 

\begin{frame}[fragile,shrink=5]{Double operators}
\begin{block}{Operator +=}
\begin{verbatim}
a += 5; // a = a + 5;
a += b * c; // a = a + b * c;
\end{verbatim}
\end{block}
\begin{block}{Operator -=}
\begin{verbatim}
a -= 3; // a = a – 3;
\end{verbatim}
\end{block}
\begin{block}{Operator *=}
\begin{verbatim}
a *= 3; // a = a * 3;
\end{verbatim}
\end{block}
\begin{block}{Operator /=}
\begin{verbatim}
a /= 3; // a = a / 3;
\end{verbatim}
\end{block}
\begin{block}{Operator \%=}
\begin{verbatim}
a %= 3; // a = a % 3;
\end{verbatim}
\end{block}
\end{frame} 

\begin{frame}[fragile]{Using variables and operators}
    \begin{exampleblock}{Example 4}
        \begin{lstlisting}
            #include <stdio.h>  
            int main() {
                int a;
                float p;
                p = 1.0 / 2.0; /* p = 0.5 */
                a = 5 / 2;     /* a = 2   */
                p = 1 / 2 + 1 / 8; /* p = 0; */
                p = 3.5 / 2.8; /* p = 1.25 */
                a = p; /* a = 1 */
                a = a + 1; /* a = 2; */
                return 0;       
            }
        \end{lstlisting}
    \end{exampleblock}
\end{frame}

\begin{frame}[fragile]{Printing to the standard output}
\begin{itemize}
\item In C doesn't exist printing command
\item Already defined function from the input/output library \texttt{stdio.h} is
used (\textbf{st}andar\textbf{d} \textbf{i}nput/\textbf{o}utput)
\large{\texttt{\#include <stdio.h>}}
\item The function used is:
\end{itemize}
\begin{verbatim}
int printf(control_array, list_of_arguments)
\end{verbatim}
\begin{itemize}
\item The control array contains any string, formating and printing arguments
with leading \% or special characters with leading \textbackslash.
\item Characters for formating are determined from the data type that
should be printed.
\end{itemize}
\end{frame}

\begin{frame}{Formatting characters}
\begin{scriptsize}
\begin{tabular}{|c|l|}
\hline \textbf{Character} & \textbf{Explanation} \\ 
\hline \texttt{\%d} & integers \\ 
\hline \texttt{\%i} & integers \\ 
\hline \texttt{\%c} & single character \\ 
\hline \texttt{\%s} & char arrays \\ 
\hline \texttt{\%e} & real number in technical foramat (е) \\
\hline \texttt{\%E} & real number in technical foramat (Е) \\ 
\hline \texttt{\%d} & real number in decimal foramat \\ 
\hline \texttt{\%f} & real number in shorter from the foramats \%е и \%f \\ 
\hline \texttt{\%g} & real number in shorter from the foramats \%Е и \%f \\  
\hline \texttt{\%u} & unsigned integer \\ 
\hline \texttt{\%o} & unsigned integer in octal \\ 
\hline \texttt{\%x} & unsigned hexadecimal integer \\
\hline \texttt{\%X} & unsigned hexadecimal integer \\ 
\hline \texttt{\%p} & pointer \\ 
\hline \texttt{\%n} & number of printed characters \\
\hline \texttt{\%\%} & printing the character \% \\ 
\hline 
\end{tabular} 
\end{scriptsize}
\end{frame}

\begin{frame}[fragile]{Function \texttt{printf} usage}
    \begin{exampleblock}{Example 5}
        \begin{lstlisting}
        #include <stdio.h>
        int main() {
           printf(" length is %d letters.\n", printf("Makedonija"));
           return 0;
        }
        \end{lstlisting}
    \end{exampleblock}
\end{frame}

\begin{frame}[fragile]{Problem 1}
    Write a program that will compute the value of the mathematical expression:
    $ x = \frac{3}{2} + (5 - \frac{46 * 5}{12})$
    \begin{exampleblock}{Solution}
        \begin{lstlisting}
        #include <stdio.h>
        int main() {
           float x = 3.0 / 2 + (5 - 46 * 5 / 12.0);
           printf("x = %.2f\n", x);
           return 0;
        }
        \end{lstlisting}
    \end{exampleblock}
\end{frame}

\begin{frame}[fragile]{Problem 2}
Write a program that for a given value of $ x $ (with the declaration
assignment) will compute and print on standard output $ x^2 $.
    \begin{exampleblock}{Solution}
        \begin{lstlisting}
        #include <stdio.h>
        int main() {
           int x = 7;
           printf("Number %d squared is: %d\n", x, x * x);
           return 0;
        }
        \end{lstlisting}
    \end{exampleblock}
\end{frame}

\begin{frame}[fragile]{Problem 3}
Write a program that for a given sides of one triangle, it will print the
perimeter and area of the square (values are \texttt{a=5, b=7.5,
c=10.2}).
    \begin{exampleblock}{Solution}
        \begin{lstlisting}
        #include <stdio.h>
        int main() {
           float a = 5;
           float b = 7.5;
           float c = 10.2;
           float L = a + b + c;
           float s = L / 2;
           float P = s * (s - a) * (s - b) * (s - c);
           printf("Perimeter is: %.2f\n", L);
           printf("Area is: %.2f\n", P);
           return 0;
        }
        \end{lstlisting}
    \end{exampleblock}
\end{frame}

\begin{frame}[fragile]{Problem 4}
Write a program for computing the arithmetic mean of the numbers 3, 5 and 12.
    \begin{exampleblock}{Solution}
        \begin{lstlisting}
        #include <stdio.h>
        int main() {
           int a = 3, b = 5, c = 12;
           float am = a + b + c / 3.0;
           printf("Arithmetic mean is: %.2f\n", as);
           return 0;
        }
        \end{lstlisting}
    \end{exampleblock}
\end{frame}

\begin{frame}[fragile]{Problem 5}
Write a program that will print the remainder from the division of number 19
with 2, 3, 5 and 8.
    \begin{exampleblock}{Solution}
        \begin{lstlisting}
        #include <stdio.h>
        int main() {
           int a = 19;
           printf("Remainder from division of %d with 2: %d\n", a, a %
           % 2);
           printf("Remainder from division of %d with 3: %d\n", a, a % 3);
           printf("Remainder from division of %d with 5: %d\n", a, a % 5);
           printf("Remainder from division of %d with 8: %d\n", a, a % 8);
           return 0;
        }
        \end{lstlisting}
    \end{exampleblock}
\end{frame}

\section{Integrated development environments (IDE)}

\begin{frame}{Integrated development environments elements}
Integrated development environment is mix of multiple programs, that are
combined in order to simplify the development process
\begin{itemize}
  \item text editor
  \item compiler
  \item debugger
  \item external library integration
  \item linker
\end{itemize}
\end{frame}

\begin{frame}{Текст уредувач (text editor)}
\begin{itemize}
  \item Програма која овозможува внесување и уредување на текстот на изворната
  програма
  \item Овозможува зачувување на програми и вчитување на веќе напишани програми
  за нивно повторно уредување 
  \item Означување на клучните зборови и команди во изворната програма (syntax
  highlighting)
\end{itemize}
\end{frame}

\begin{frame}{Преведувач (compiler)}
\begin{itemize}
  \item Ја преобразува (преведува) изворната програма од јазикот за програмирање
  во кој е напишана во јазик разбирлив за компјутерот
  \item Се разликуваат два вида преведувачи: \textbf{интерпретери} и \textbf{компајлери} 
  \item Интерпретер е преведувач кој ја \emph{обработува одделно секоја команда}, ја
  проверува за грешки и ја извршува, по што поминува на следната команда  итн.
  \item Компајлер е преведувач кој ја \emph{обработува целата програма}, ја проверува
  за грешки и ја преведува, по што се добива извршната програма.
  \begin{itemize}
  \item Така добиената извршна програма може да се извршува
  \end{itemize}
\end{itemize}
\end{frame}

\begin{frame}{Дебагер (debugger)}
\begin{itemize}
  \item Компајлерите и интерпретерите ги откриваат грешките (синтаксички) во
  програмата поради не правилно користење на јазикот за програмирање
  \item Друг вид на грешки се логичките грешки
  \begin{itemize}
  \item Програмата не го прави тоа за кое што е наменета 
  \item Се откриваат многу тешко
  \end{itemize}
  \item Дебагер е програма која помага при барање на логичките грешки
  \begin{itemize}
  \item Овозможува следење на извршувањето на програмата чекор по чекор
  \end{itemize}
\end{itemize}
\end{frame}

\begin{frame}{Интеграција на библиотеки со функции}
\begin{itemize}
  \item Интегрирање и користење на претходно создадени и проверени модули (потпрограми), уште наречени и функции
  \item Ваквиот начин на организација на програмите има голем број на предности
  \item Повторно искористување на готови функционалности
  \item Пример библиотеки
  \begin{itemize}
    \item За управување со стандардниот влез и излез
    \item За стандардни математички операции 
  \end{itemize}
\end{itemize}
\end{frame}

\begin{frame}{Поврзувач (linker)}
\begin{itemize}
  \item Понекогаш програмата е премногу голема за да се напише во една датотека
  \begin{itemize}
    \item различните делови може да се пишуваат од различни програмери.
    \item некои делови од дадена програма можат да бидат искористени и во друга програма 
    \item Одделно компајлираните делови е неопходно да бидат обединети во една
    цела извршна програма со помош на \textbf{поврзувачот}
    \item Друга улога на поврзувачот е да ги „сврзе“ со програмата потребните
     библиотеки со стандардните функции
  \end{itemize}
\end{itemize}

\end{frame}

\begin{frame}{Околини за развој}{(ntegrated Development Environment - IDE}
\begin{itemize}
  \item Сите овие елементи на околината за развој се обединуваат (интегрираат)
  во т.н. интегрирани околини за развој
  \item Пример за IDE е околината која ќе се користи на овој курс, Code::Blocks
\end{itemize}
\begin{center}
\includegraphics[scale=0.5]{images/cb_logo}
\end{center}
\end{frame}

\section{Code::Blocks - инсталација}
\begin{frame}{Code::Blocks - инсталација}
\begin{itemize}
  \item Како да го најдеме и инсталираме Code::Blocks
  \item Code::Blocks е \textbf{слободен софтвер} и може да се најде
  на\linebreak
  \href{http://www.codeblocks.org/downloads}{http://www.codeblocks.org/downloads}
  \item Во централниот дел на страната има три линка: \textbf{Download the binary release}, Download the source code и Retrieve source code from SVN
  \item За наједноставна инсталација се препорачува да се избере првиот линк -
  \textbf{Download the binary release},
\end{itemize}
\end{frame}

\begin{frame}{Code::Blocks – инсталација (2)}
\begin{itemize}
  \item За почетниците се препорачува да ја симнат верзијата што во нејзе
  вклучува \textbf{MinGW} setup
    \begin{itemize}
  \item моментално тоа е линкот \textbf{codeblocks-10.05mingw-setup.exe} кој е наменет за
  корисниците на сите \textbf{Windows} оперативни системи
  \item Со клик на изворот Sourceforge.net се отвора нова страницата која по
  истекот на 5 секунди сама ќе ви понуди опција да ја зачувате датотеката на од вас избрана локација
  \item По зачувувањето на датотеката следете ги инструкциите за инсталирање
    \end{itemize}
\end{itemize}
\end{frame}

\begin{frame}{Code::Blocks – главен прозорец}
\begin{center}
\includegraphics[scale=0.26]{images/cb_main}
\end{center}
\end{frame}

\begin{frame}{Елементи на главниот прозорец}
\begin{itemize}
  \item Лента со менија
    \begin{itemize}
      \item лентата со менија се наоѓа во најгорниот дел на прозорецот, веднаш под неговиот насловот 
      \item Во неа се наоѓаат менијата File, Edit, View, Search, Project, Build,
      Debug, wxSmith, Tools, Plugins, Settings, Help
    \end{itemize}
    \item Лента со алатки
    \begin{itemize}
      \item лентите со алатки (копчиња за стартување на најчесто користените
      команди на околината) се наоѓаат непосредно под лентата со паѓачки менија
    \end{itemize}
    \item Работна површина
    \begin{itemize}
      \item Потпрозорец за уредувачот на текст
      \item Прозорец за соопштенија.
      \item Прозорец за организација на работата на програмата
    \end{itemize}   
\end{itemize}
\end{frame}

\begin{frame}{Програмирање во C со Code::Blocks}{Креирање проект}
\begin{enumerate}
  \item Стартувајте CodeBlocks
  \item File -> New -> Project -> Empty Project -> Go 
\end{enumerate}
\begin{center}
\includegraphics[scale=0.3]{images/cb_new}
\end{center}
\end{frame}

\begin{frame}{Програмирање во C со Code::Blocks}{Креирање проект}
\begin{enumerate}
\setcounter{enumi}{2}
  \item Одберете  GNU GCC Compiler
  \item Изберете ги следните 2 опции ако сакате да креирате “debug” и “release”
  configuration
\end{enumerate}
\begin{center}
\includegraphics[scale=0.3]{images/cb_compiler}
\end{center}
\end{frame}

\begin{frame}{Додавање на изворна датотека}
\begin{enumerate}
\setcounter{enumi}{4}
  \item Додадете изворна датотека во проектот: File -> New -> File -> C/C++
  Source
  \item Одберете C како програмски јазик 
  \item Внесете го името на датотеката со полната патека и не заборавајте да го
  вклучите  "Add file to active project"
\end{enumerate}
\begin{center}
\includegraphics[scale=0.25]{images/cb_source}
\includegraphics[scale=0.4]{images/cb_include}
\end{center}
\end{frame}

\begin{frame}{Програмирање}
\begin{itemize}
  \item За секој проект може да се постават следните опции "Project  Build Options..
Compiler Flags"
  \item За изградба на проектот (build) притиснете Ctrl + F9 
  \item За извршување на проектот притиснете Ctrl + F10
\end{itemize}
\begin{center}
\includegraphics[scale=0.25]{images/cb_run}
\includegraphics[scale=0.1]{images/cb_flags}
\end{center}
\end{frame}

\begin{frame}{Задачи за дома}
\begin{itemize}
  \item Во продолжение се наведени неколку задачи кои би требало да се обидете
  да ги изработите дома
  \item Со нивна изработка ќе бидете подготвени за успешна работа на
  претстојните лабораториски вежби
\end{itemize}
\end{frame}

\begin{frame}[fragile]{Задача 1}
Обидете се да креирате нов проект со една .с датотека и во неа внесете го
текстот на следнава програма:
\begin{lstlisting}
#include <stdio.h>

int main() {
    printf("Zdravo, kako si?\n");
    return 0;
}
\end{lstlisting}

\end{frame}

\begin{frame}{Задача 1}
\begin{itemize}
  \item Извршете ја програмата
\begin{itemize}
  \item Што добивате како резултат?
\end{itemize}
  \item Доколку сте направиле грешка при пишувањето на текстот поправете и
  извршете уште еднаш. 
  \item Направете намерно некоја грешка во текстот. Извршете повторно!
\begin{itemize}
  \item Што се случува сега?
\end{itemize}
\end{itemize}
\end{frame}

\begin{frame}[fragile]{Задача 2}
\definecolor{light-gray}{gray}{0.90}
Во текстот на програмата додадете го означениот ред:
\begin{lstlisting}
#include <stdio.h>
int main() {
    printf("Zdravo, kako si?\n");
    printf("Neshto ne ti se pravi muabet?\n");
    return 0;
}
\end{lstlisting}
Кој е резултатот од извршувањето сега?
\end{frame}
