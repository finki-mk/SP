%%%%%%%%%%%%%%%%%%%%%%%%%%%%%%%%%%%%%%%%%
%%%%%%%%%% Content starts here %%%%%%%%%%
%%%%%%%%%%%%%%%%%%%%%%%%%%%%%%%%%%%%%%%%%

\begin{frame}[fragile]{Opening files for reading/writing}{Remainders from
lectures}

\begin{itemize}
  \item Processing files includes writing, reading and changing
  contents of files to some standard media as disk.
  \item Processing files in C is done using the struct \texttt{FILE} defined in
  \texttt{stdio.h}
  \item To start processing the file first it must be open using the function
  \texttt{fopen} that returns pointer to the struct (\texttt{FILE*})
\end{itemize}
\begin{exampleblock}{Function for opening file}
\begin{lstlisting}
FILE *fopen(const char *filename, const char *mode);
\end{lstlisting}
\begin{scriptsize}
\texttt{filename} - full path of the file we want to open\\
\texttt{mode} - opening mode
\end{scriptsize}
\end{exampleblock}
\end{frame}

\begin{frame}{Modes of opening files}{Remainders from
lectures}
\begin{tabular}{| c | p{.8\textwidth} |}
\hline
\textbf{Mode} & \textbf{Meaning}\\
\hline
\texttt{r}  &  Opens existing file only for reading\\
\hline
\texttt{w} & Opens existing file for writing (the file must exist)\\
\hline
 \texttt{a} & Opens existing file for appending (the file must exist)\\
 \hline
 \texttt{r+} & Opens file for reading and writing at the
 beginning of the file\\
 \hline
 \texttt{w+} & Opens file for reading and writing (deletes the contents of the
 file)\\
 \hline
 \texttt{a+} & Opens file for reading and writing (appends at the end of
 the file if it exists)\\
 \hline
\end{tabular}
\end{frame}

\begin{frame}[fragile]{Opening files for reading/writing}{Remainders from
lectures}

\begin{exampleblock}{Example opening file}
\begin{lstlisting}
FILE *fp = fopen("C:\\test.txt", "r");
\end{lstlisting}
\begin{scriptsize}
To open in binary mode \texttt{b} should be appended at the mode of opening ex.
(\texttt{``rb''})
\end{scriptsize}
\end{exampleblock}
After the end of processing, the file should be closed using the function \texttt{fclose}
\begin{exampleblock}{Example of closing file}
\begin{lstlisting}
fclose(fp);
\end{lstlisting}
\end{exampleblock}
\end{frame}

\begin{frame}[fragile]{Reading and writing from/to
file}{Remainders from
lectures} 
Functions for reading from file
\begin{lstlisting}
int fscanf(FILE* fp, ``control array'', arguments_list)
int fgetc(FILE* fp)
\end{lstlisting}

Functions for writing to file
\begin{lstlisting}
int fprtinf(FILE* fp, ``control array'', arguments_list)
int fputc(int c, FILE* fp)
\end{lstlisting}

\end{frame}


\begin{frame}[fragile]{Problem 1}{Solution 1/2}
Write a program that for given textual file will find the ratio of
vowel/consonants. The name of the file is passed as argument.
\lstinputlisting[lastline=19]{src/av10/p1.c}
\end{frame}

\begin{frame}[fragile]{Problem 1}{Solution 2/2}
\lstinputlisting[firstline=21]{src/av10/p1.c}
\end{frame}

\begin{frame}{Problem 2}
Write a program that will copy each line from given text file in to new output
file, but before the line will print the length of the line. The name of the
input and output file are given as command line arguments, and if they are
missing write correct usage. Each line can have max of 80 characters.
\end{frame}

\begin{frame}[fragile]{Problem 2}{Solution 1/2} 
\lstinputlisting[lastline=20]{src/av10/p2.c}
\end{frame}



\begin{frame}[fragile]{Problem 2}{Solution 2/2} 
\lstinputlisting[firstline=21]{src/av10/p2.c}
\end{frame}

\begin{frame}{Problem 3}
Write a program that will read elements of a matrix written in text file with
name ``matrica.txt''. In the first line of the file are written the number of
rows and columns of the matrix. Each element of the matrix is floating point
number written in separate line. The transposed matrix write in a new output
file ``matrica2.txt'' using the same format.
\end{frame}

\begin{frame}[fragile]{Problem 3}{Solution 1/2} 
\lstinputlisting[lastline=22]{src/av10/p3.c}
\end{frame}



\begin{frame}[fragile]{Problem 3}{Solution 2/2} 
\lstinputlisting[firstline=23]{src/av10/p3.c}
\end{frame}


\begin{frame}{Problem 4}
Given a text file ``KRSPrimer.txt'' write a program that will print the count of
lines that have at least 10 vowels, and the total vowels in the file. Each line
has max of 80 characters.
\end{frame}

\begin{frame}[fragile]{Problem 4}{Solution}
\lstinputlisting{src/av10/p4.c}
\end{frame}

\begin{frame}{Problem 5}
Write a program that from a given text file will print the words that have
multiple occurance of the same letter (the letter occurs two or more times).
Ignore the case of the letters in the comparison. Each word is written in a
separate line in the file and the max length of the words is 20 characters. The
name of the file is a command line argument.
\begin{exampleblock}{Example words}
banana, text, different, Copacabana, etc\ldots
\end{exampleblock}
\end{frame}

\begin{frame}[fragile]{Problem 5}{Solution 1/2}
\lstinputlisting[lastline=22]{src/av10/p5.c}
\end{frame}

\begin{frame}[fragile]{Problem 5}{Solution 2/2}
\lstinputlisting[firstline=23]{src/av10/p5.c}
\end{frame}

\begin{frame}{Problem 6}
Write a program that will print the count of occurances of a given word
composed only from digits in a given textual file. The name of the file and the
word to be found are given as a command line arguments. The program should check
if all needed arguments are provided, and if not so should print apropriate
message. 
\end{frame}

\begin{frame}[fragile]{Problem 6}{Solution 1/2}
\lstinputlisting[lastline=16]{src/av10/p6.c}
\end{frame}

\begin{frame}[fragile]{Problem 6}{Solution 2/2}
\lstinputlisting[firstline=17]{src/av10/p6.c}
\end{frame}

