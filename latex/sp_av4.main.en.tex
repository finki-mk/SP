
%%%%%%%%%%%%%%%%%%%%%%%%%%%%%%%%%%%%%%%%%
%%%%%%%%%% Content starts here %%%%%%%%%%
%%%%%%%%%%%%%%%%%%%%%%%%%%%%%%%%%%%%%%%%%

\section{Loops}
\begin{frame}[fragile]{Problem 1а}
Write a program that computes the sum of all even twodigit numbers. Print the
result on standard output.
\pause
\begin{exampleblock}{Solution}
\lstinputlisting{src/av4/p1a.c}
\end{exampleblock}
\end{frame}


\begin{frame}[fragile]{Problem 1б}
\begin{scriptsize}
Write a program that computes the sum of all odd twodigit numbers. Print the
result on standard output in the following format:	
\texttt{11 + 13 + 15 + 17 + ... + 97 + 99 = 2475}\\
\textbf{Note: The program should be written without the usage of if}
\end{scriptsize}
\pause
\begin{columns}
\column{.5\textwidth}
\begin{exampleblock}{Solution - Version 1}
\lstinputlisting{src/av4/p1b1.c}
\end{exampleblock}
\pause
\column{.5\textwidth}
\begin{exampleblock}{Solution - Version 2}
\lstinputlisting{src/av4/p1b2.c}
\end{exampleblock}
\end{columns}
\end{frame}

\begin{frame}[fragile]{Problem 2}{Solution with usage of \texttt{while} and
\texttt{do\ldots while}} 
Write a program that computes $y = x^n$ for given natural number $n, n>=1$ and
real number $x$.
\pause
\begin{columns}
\column{.5\textwidth}
\begin{exampleblock}{Solution - with usage of \texttt{while}}
\lstinputlisting{src/av4/p2a.c}
\end{exampleblock}
\pause
\column{.5\textwidth}
\begin{exampleblock}{Solution - со употреба на \texttt{do...while}}
\lstinputlisting{src/av4/p2b.c}
\end{exampleblock}
\end{columns}
\end{frame}

\begin{frame}[fragile]{Problem 2}{Solution со употреба на \texttt{for}}
Write a program that computes $y = x^n$ for given natural number $n, n>=1$ and
real number $x$.
\pause
\begin{exampleblock}{Solution со употреба на \texttt{for}}
\lstinputlisting{src/av4/p2c.c}
\end{exampleblock}
\end{frame}


\begin{frame}[fragile]{Problem 3}
Write a program that for n numbers read from SI will count the numbers divisible
by 3, have residue 1, and have residue 2.\\
\textbf{Note: Solve by using while, do \ldots while and for}
\end{frame}


\begin{frame}[fragile]{Solution 3}{\texttt{while}}
\begin{exampleblock}{Solution by using \texttt{while}}
\lstinputlisting{src/av4/p3a.c}
\end{exampleblock}
\end{frame}


\begin{frame}[fragile]{Solution 3}{\texttt{do while}}
\begin{exampleblock}{Solution by using \texttt{do\ldots while}}
\lstinputlisting{src/av4/p3b.c}
\end{exampleblock}
\end{frame}


\begin{frame}[fragile]{Solution 3}{\texttt{for}}
\begin{exampleblock}{Solution by using \texttt{for}}
\lstinputlisting{src/av4/p3c.c}
\end{exampleblock}
\end{frame}

\begin{frame}[fragile]{Problem 4}
Write a program that will print all 4-digit numbers in which the sum of the
three least significant digits is equal to the most significant digit.
\begin{exampleblock}{Exmple}
	\texttt{4031 (4=0+3+1), 5131 (5=1+3+1)}
\end{exampleblock}
\end{frame}

\begin{frame}[fragile]{Solution 4}
\begin{exampleblock}{Solution}
\lstinputlisting{src/av4/p4.c}
\end{exampleblock}
\end{frame}

\begin{frame}[fragile]{Problem 5}
Write a program that will print all numbers in given range which are read the
same from left to right and oposite.
\begin{exampleblock}{Example}
\texttt{12345    54321}
\end{exampleblock}
\end{frame}

\begin{frame}[fragile]{Solution 5}
\begin{exampleblock}{Solution}
\lstinputlisting{src/av4/p5.c}
\end{exampleblock}
\end{frame}


\begin{frame}[fragile]{Problem 6}
\scriptsize{
Write a program that for unknown count of integers read from SI will find the
number with maximum value. The program stops when the reading of integer fails.}
\pause
\begin{exampleblock}{Solution}
\lstinputlisting{src/av4/p6.c}
\end{exampleblock}
\end{frame}


\begin{frame}[fragile]{Problem 7}
Write a program that for unknown count of integers read from SI will find the
number with maximum value. Numbers larger than 100 should be ignored. The
program stops when the reading of integer fails.
\end{frame}

\begin{frame}[fragile]{Solution 7}
\begin{exampleblock}{Solution}
\lstinputlisting{src/av4/p7.c}
\end{exampleblock}
\end{frame}

\begin{frame}{Problem 8}
Write a program that for unknown count of integers read from SI will find the
two with maximum values. 
\begin{exampleblock}{Example}
For numbers \texttt{2 4 7 4 2 1 8 6 9 7 10 3} the program should print
\texttt{10} and \texttt{9}.
\end{exampleblock}
\end{frame}

\begin{frame}[fragile]{Solution 8}
\begin{exampleblock}{Solution}
\lstinputlisting{src/av4/p8.c}
\end{exampleblock}
\end{frame}

\begin{frame}{Problem 9}
Write a program that for N integers read from SI will the difference of the sums
of numbers on odd and even positions (by the order of reading). If this
difference is less than 10 print the message ``The two sums are close'' else
print ``The two sums are far''.
\begin{exampleblock}{Example}
For the numbers:\\
\texttt{{\color{red}2} 4 {\color{red}3} 4 {\color{red}2} 1 {\color{red}1} 6
{\color{red} 1} 7}\\
\texttt{{\color{red} suma\_odd\_positions = 9}}\\
\texttt{suma\_even\_positions = 22}\\
Will print: \texttt{The two sums are far}
\end{exampleblock}
\end{frame}

\begin{frame}[fragile]{Solution 9}
\begin{exampleblock}{Solution}
\lstinputlisting{src/av4/p9.c}
\end{exampleblock}
\end{frame}

\begin{frame}{Problem 10}
Write a program that for unknown count of integers read from SI will find the
positions of the successive numbers with maximum sum. The program stops when two
successive read numbers are negative.
\end{frame}

\begin{frame}[fragile]{Solution 10}
\begin{exampleblock}{Solution}
\lstinputlisting{src/av4/p10.c}
\end{exampleblock}
\end{frame}

\section{The \texttt{switch} expession}
\begin{frame}{Problem 1}
Write a program that will enable transforming of twodigit numbers in the
following way:\\
For the twodigit number 89 should print ``eighty-nine''.
\end{frame}

\begin{frame}[t,fragile,shrink=35]{Solution 1}
\begin{columns}
\column{.5\textwidth}
\begin{exampleblock}{Solution first part}
\lstinputlisting[lastline=40]{src/av4/p11.c}
\end{exampleblock}
\column{.5\textwidth}
\begin{exampleblock}{Solution seccond part}
\lstinputlisting[firstline=41]{src/av4/p11.c}
\end{exampleblock}
\end{columns}
\end{frame}


\begin{frame}{Problem 2}
Write a program for simple calculator. The program should read two numbers and
operator in the following format:\\
\texttt{num1 operator num2}\\
After the operation depending on the sign, the result is printed in the
following format:\\
\texttt{num1 operator num2 = result}
\end{frame}

\begin{frame}[fragile,shrink=10]{Solution 2}
\begin{exampleblock}{Solution}
\lstinputlisting{src/av4/p12.c}
\end{exampleblock}
\end{frame}

