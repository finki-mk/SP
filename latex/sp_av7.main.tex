
%%%%%%%%%%%%%%%%%%%%%%%%%%%%%%%%%%%%%%%%%
%%%%%%%%%% Content starts here %%%%%%%%%%
%%%%%%%%%%%%%%%%%%%%%%%%%%%%%%%%%%%%%%%%%


\begin{frame}[fragile]{Рекурзивни функции}{Потсетување од предавања}
\begin{itemize}
    \item Функциите во C може да повикуваат други функции
    \begin{itemize}
        \item Ова повикување може да оди во произволна длабочина
    \end{itemize}    
    \item Една функција може да се повикува самата себе
    \item Таквиот начин на повикување се нарекува \textbf{рекурзија}    
\end{itemize}
\begin{exampleblock}{Пример на рекурзивна функција}
\begin{lstlisting}
int faktoriel(int n) {
    if(n == 0) return 1;
    else return n * faktoriel(n - 1);
}
\end{lstlisting}
\end{exampleblock}
\end{frame}

\begin{frame}{Задачa 1}
Да се пресмета збирот:\\
\texttt{1!+(1+2)!+(1+2+3)!+\ldots+(1+2+...+n)!}
\\Овој пат:\\
\begin{itemize}
    \item Користете \textbf{рекурзивна} функција за пресметување на збирот на првите k
    природни броеви
    \item Користете \textbf{рекурзивна} функција за пресметување факториел на еден
    природен број k
\end{itemize}
\end{frame}

\begin{frame}[fragile]{Задача 1}{Решение}
\lstinputlisting[lastline=14]{src/av7/p1.c}
\end{frame}

\begin{frame}[fragile]{Задача 1}{Решение}
\lstinputlisting[firstline=15]{src/av7/p1.c}
\end{frame}


\begin{frame}{Задачa 2}
Да се напише програма која за даден природен број ја пресметува разликата помеѓу
најблискиот поголем од него прост број и самиот тој број. Програмата треба да
користи \textbf{рекурзивна} функција за наоѓање на соодветниот прост број.
\begin{exampleblock}{Пример}
Ако се внесе \texttt{573}, програмата треба да испечати\\
\texttt{577 – 573 = 4}
\end{exampleblock}
\end{frame}

\begin{frame}[fragile]{Задачa 2}{Решение} 
\lstinputlisting{src/av7/p2.c}
\end{frame}

\begin{frame}{Задачa 3}
Да се напише програма што ќе ја испишува вредноста на n-тиот член на низата дефинирана со:
\[
   \begin{array}{l}
   x_1 = 1\\
   x_2 = 2\\ 
   \vdots\\
   x_n = \frac{n - 1}{n}x_{n - 1} + \frac{1}{n}x_{n - 2}
   \end{array}
\]
\end{frame}


\begin{frame}[fragile]{Задачa 3}{Решение}
\lstinputlisting{src/av7/p3.c}
\end{frame}


\begin{frame}[fragile]{Задачa 4}
Да се напише рекурзивна функција која ќе го пресметува збирот на цифрите на еден
број.
\begin{exampleblock}{Пример}
\begin{verbatim}
sumDigits(126) -> 9
sumDigits(49) -> 13
sumDigits(12) -> 3
\end{verbatim}
\end{exampleblock}
\pause
\lstinputlisting{src/av7/p4.c}
\end{frame}

\begin{frame}[fragile]{Задача 5}
За даден број n, да се напише рекурзивна функција која ќе ги изброи појавувањата
на цифрата 8, со тоа што ако се појави цифра 8 со уште една цифра 8 веднаш лево
од неа таа се брои двојно, така 8818 дава 4.
\begin{exampleblock}{Пример}
\begin{verbatim}
count8(8) -> 1
count8(818) -> 2
count8(8818) -> 4
\end{verbatim}
\end{exampleblock}
\pause
\lstinputlisting{src/av7/p4.c}
\end{frame}


\begin{frame}{Задачa 6}
Да се напише програма која што за дадена низа од природни броеви  која што се
внесува од тастатура) ќе го отпечати најголемиот заеднички делител (НЗД) на
нејзините елементи.  Програмата задолжително треба да содржи рекурзивна функција
за пресметување на НЗД на два природни броја.
\begin{exampleblock}{Пример}
\texttt{48 36 120 72 84}\\
На екран треба да се отпечати:\\
\texttt{NZD na elementite na ovaa niza e 12}
\end{exampleblock}
\end{frame}

\begin{frame}[fragile,shrink=5]{Евклидов алгоритам}{Задачa 6}
\begin{itemize}
  \item НЗД за два броја може да се пресмета со користење на Евклидовиот
  алгоритам
  \item За да се пресмета НЗД за броевите m и n, се пресметува остатокот при
  делење на m со n
  \begin{itemize}
  \item Ако остатокот не е 0, се пресметува остатокот при делење на n со m \% n
  \item Постапката се повторува сé додека се добиваат ненулти остатоци
  \item Ако остатокот е 0, НЗД за двата броја е последниот пресметан ненулти
  остаток
  \end{itemize}
\end{itemize}
\begin{exampleblock}{Пример}
\begin{verbatim}
НЗД(20, 12)
20 % 12 = 8
12 % 8 = 4
8 % 4 = 0
НЗД(20, 12) = 4
\end{verbatim}
\end{exampleblock}
\end{frame}

\begin{frame}[fragile]{Задача 6}{Решение}
\lstinputlisting{src/av7/p6.c}
\end{frame}

\begin{frame}[fragile]{Задача 7}{За дома}
Да се напише програма која што за дадена низа од природни броеви (која што се
внесува од тастатура) ќе го отпечати најмалиот заеднички содржател (НЗС) на
нејзините елементи.  Програмата задолжително треба да содржи рекурзивна функција
за пресметување на НЗС на два природни броја.
\begin{exampleblock}{Пример}
\begin{verbatim}
За низата:
18 12 24 36 6
На екран треба да се отпечати:
NZS na elementite na ovaa niza e 72
\end{verbatim}
\end{exampleblock}
\end{frame}
   

\begin{frame}[fragile]{Задача 8}
Да се напише програма која за дадена низа од цели броеви (која што се
внесува од тастатура) ќе го отпечати најмалиот елемент. Програмата треба да
содржи рекурзивна функција за наоѓање на најмал елемент во дадена низа.
\begin{exampleblock}{Пример}
\begin{verbatim}
За низата:
5 8 3 12 9 6
На екран треба да се отпечати:
Najmal element vo nizata e 3
\end{verbatim}
\end{exampleblock}
\end{frame}

\begin{frame}[fragile]{Задачa 8}{Решение}
\lstinputlisting{src/av7/p8.c}
\end{frame}

