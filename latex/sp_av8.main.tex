%%%%%%%%%%%%%%%%%%%%%%%%%%%%%%%%%%%%%%%%%
%%%%%%%%%% Content starts here %%%%%%%%%%
%%%%%%%%%%%%%%%%%%%%%%%%%%%%%%%%%%%%%%%%%


\begin{frame}[fragile,shrink=10]{Покажувачи (Pointers)}{Потсетување од
предавања}
  \begin{block}{Што е покажувач (pointer)?}
    \textbf{Покажувач} е \alert{податочен тип} кој што чува (покажува кон)
    некоја мемориска локација. Оваа мемориска локација се чува преку нејзината адреса (некаков
    број).
  \end{block}
  \begin{block}{Како се декларира покажувач?}
  
    \texttt{pointer\_type *name;}\\
    Пример:\\
    \texttt{float *p;}\\
    \alert{Вака декларираниот покажувач не покажува никаде!}
  \end{block}
  \begin{block}{Зошто служат покажувачите?}
    \begin{itemize}
      \item За брзо и ефикасно изминување на сложени податочни структури како
      низи и дрва, \ldots
      \item За ефикасно пренесување на сложени аргументи во функции. Пренесување
      на низа, структура и слично
      \item За пренесување на аргументи чии што вредности сакаме да останат во
      онаа состојба во која се наоѓаат по извршувањето на функцијата
    \end{itemize}
  \end{block}
  
\end{frame}

\begin{frame}{Задачa 1}
Да се напише функција која за низа од N цели броеви ќе ги пронајде почетокот и
должината на најголемата растечка подниза.
\begin{exampleblock}{Пример}
За низата\\
\texttt{2 3 {\color{red}1 4 7 12} 7 9 1}\\
ќе врати \texttt{2 4}
\end{exampleblock}
\end{frame}

\begin{frame}[fragile]{Задача 1}{Решение}
\begin{columns}
\column{.5\textwidth}
\lstinputlisting[lastline=23]{src/av8/p1.c}
\column{.5\textwidth}
\lstinputlisting[firstline=25]{src/av8/p1.c}
\end{columns}
\end{frame}

\begin{frame}{Задачa 2}
Да се напише фунција која влезната низа $a_0, b_1, \ldots, b_{n-1}$ ќе ја
трансформира во излезна низа $b_0, b_1, \ldots, b_{n-1}$ на следниот
начин
\[
\begin{array}{l}
b_0 = a_0 + a_{n-1}\\
b_1 = a_1 + a_{n-2}\\
\vdots\\
b_n = a_{n-1} + a_{0}
\end{array}
\]
\begin{exampleblock}{Пример} 
Влезната низа\\
\texttt{1 2 3 5 7}\\
треба да се трансформира во\\
\texttt{8 7 6 7 8}
\end{exampleblock}
\end{frame}

\begin{frame}[fragile]{Задачa 2}{Решение} 
\lstinputlisting{src/av8/p2.c}
\end{frame}

\begin{frame}{Задачa 3}
Да се напишат следните функции за пребарување во низа:
\begin{itemize}
  \item Линеарно пребарување
  \item Бинарно пребарување
\end{itemize}  
Потоа да се напише главна програма во која ќе се пополнува низа со
броевите од 1 до 1 000 000, а потоа се генерира случаен број во овој опсег чија
што позиција треба да се пронајде со повикување на двете функции за
пребарување.\\
\textbf{За дома:} За двете функции избројте го и споредето го бројот на потребни
итерации за проаноѓање на бројот.
\end{frame}


\begin{frame}[fragile]{Задачa 3}{Решение}
\lstinputlisting[firstline=6,lastline=23]{src/av8/p3.c}
\textbf{За дома:} Да се напише рекурзивна фунција за бинарно пребарување
\end{frame}

\begin{frame}[fragile]{Задачa 3}{Решение}
\lstinputlisting[firstline=26]{src/av8/p3.c}
\end{frame}


\begin{frame}{Задачa 4}
Да се напишат фунции за сортирање на низа со помош на следните методи за
сортирање:
\begin{itemize}
  \item Метод на меурче (Bubble sort)
  \item Метод со избор на елемент (Selection sort)
  \item Метод со вметнување (Insertion sort)
\end{itemize}
Да се напишат функции за внесување и печатење на елементите на една низа и да се
напише главна програма во која се тестираат сите методи за сортирање.
\end{frame}

\begin{frame}[fragile]{Bubble sort}{Задачa 4}
\begin{scriptsize}
Се започнува од првиот елемент и се споредуваат секои два соседни додека не се
дојде до последниот елемент. При секое споредување, ако претходниот има поголема
вредност, тогаш си ги заменуваат местата. Така најголемиот елемент се доведува
на последната позиција во низата. Се повторува истата постапка од 1-от до
претпоследниот елемент во низата, така што сега на претпоследната позиција ќе
исплива елемент помал од најголемиот елемент во низата итн. На крајот се
споредуваат само 1-от и 2-от елемент од низата.
\end{scriptsize}
\lstinputlisting[firstline=3, lastline=11]{src/av8/p4.c}
\end{frame}

\begin{frame}[fragile]{Selection sort}{Задачa 4}
\begin{scriptsize}
Се  пронаоѓа најмалиот елемент во низата и истиот се заменува со првиот елемент.
Потоа, првиот елемент на низата се игнорира (бидејќи се знае дека тој е најмал)
и рекурзивно се сортира преостанатата подниза (од вториот елемент, па до
крајот). Постапката се повторува сé дури не остане само еден елемент. Тоа е
граничниот случај – се престанува со сортирањето
\end{scriptsize}
\lstinputlisting[firstline=13, lastline=28]{src/av8/p4.c}
\end{frame}

\begin{frame}[fragile]{Insertion sort}{Задачa 4}
\begin{scriptsize}
Со овој метод се сортира на тој начин што секој елемент се вметнува на
соодветната позиција, од што и доаѓа самото име. Во првата итерација, вториот
елемент a[1] се споредува со првиот елемент a[0]. Во втората итерација третиот
елемент се споредува со првиот и вториот. Генерално, во секоја итерација
елементот се споредува со сите елементи пред него. Ако при споредбата се покаже
дека тој елемент треба да се вметне на соодветната позиција, тогаш се создава
простор со поместување на сите елементи десно од тој елемент за еден и се
вметнува елементот. Оваа процедура се повторува за секој елемент во низата. 
\end{scriptsize}
\lstinputlisting[firstline=30, lastline=41]{src/av8/p4.c}
\end{frame}

\begin{frame}[fragile]{Задачa 4}{Решение}
\lstinputlisting[firstline=43]{src/av8/p4.c}
\end{frame}
