%%%%%%%%%%%%%%%%%%%%%%%%%%%%%%%%%%%%%%%%%
%%%%%%%%%% Content starts here %%%%%%%%%%
%%%%%%%%%%%%%%%%%%%%%%%%%%%%%%%%%%%%%%%%%


\begin{frame}{String functions}{\texttt{<string.h>}}
String mutable functions
\begin{itemize}
  \item \texttt{strcpy} - copy one string to another
  \item \texttt{strncpy} - copy n bytes from one string to another, from
  \texttt{src} are copied or \texttt{nulls} are paded
  \item \texttt{strcat} - concatenates string at the end of another
  \item \texttt{strncat} - concatenates n bytes from one to another
\end{itemize}
Memory changing functions
\begin{itemize}
  \item \texttt{memset} - filles array with some byte
\end{itemize}
Functions for conversion string to numbers
\begin{itemize}
  \item \texttt{atof} - converts string to decimal number
  \item \texttt{atoi} - converts string to integer number
\end{itemize}

\end{frame}

\begin{frame}[shrink=10]{String functions}{\texttt{<string.h>}}
String checking functions
\begin{itemize}
  \item \texttt{strlen} - return the length of the string
  \item \texttt{strcmp} - compares two strings
  \item \texttt{strncmp} - compares n bytes from two strings
  \item \texttt{strchr} - finds the first occurance of given character in some
  string
  \item \texttt{strrchr} - finds the last occurance of given character
  \item \texttt{strspn} - finds the first occurance of given character in some
  string not in a given set of chars
  \item \texttt{strcspn} - finds the last occurance of given character not in a
  set of chars
  \item \texttt{strpbrk} - finds the first occurance of given character in some
  string in a given set of chars
  \item \texttt{strstr} - finds the first occurance of string in other string
  \item \texttt{strtok} - finds in string the next token
\end{itemize} 
\end{frame}

\begin{frame}[shrink=10]{Functions for characters}{\texttt{<ctype.h>}}
\begin{itemize}
  \item \texttt{isalnum} - checks if the character is alphanumeric (letter or
  number)
  \item \texttt{isalpha} - checks if the character is letter
  \item \texttt{iscntrl} - checks if the character is control
  \item \texttt{isdigit} - checks if the character is digit
  \item \texttt{isxdigit} - checks if the character is hex digit
  \item \texttt{isprint} - checks if the character is printable
  \item \texttt{ispunct} - checks if the character is interpunction
  \item \texttt{isspace} - checks if the character is space (blank)
  \item \texttt{islower} - checks if the character is lowercase
  \item \texttt{isupper} - checks if the character is uppercase
  \item \texttt{tolower} - converts to lowercase
  \item \texttt{toupper}  - converts to uppercase
  \item \texttt{isgraph} - checks if the character has local graphic
  representation
\end{itemize}
\end{frame}




\begin{frame}{Problem 1}
Write a function that will find how many times a given character occures in
given string.
\begin{exampleblock}{Example}
For the string\\
\texttt{``hello FINKI''}\\
character \texttt{'l'} occures 2 times
\end{exampleblock}
\end{frame}

\begin{frame}[fragile]{Problem 1}{Solution}
\lstinputlisting{src/av9/p1.c}
\end{frame}

\begin{frame}{Problem 2}
Write a function that will return the length of a string. Also write a recursive
solution.
\begin{exampleblock}{Example} 
For string: \texttt{``zdravo!''}\\
The result should be: 7
\end{exampleblock}
\end{frame}

\begin{frame}[fragile]{Problem 2}{Solution} 
\lstinputlisting{src/av9/p2.c}
\end{frame}

\begin{frame}{Problem 3}
Write a program that will print substring from given string, determined with the
position and the length as parameters read from SI. The substring starts from
the character on the position counted from left to right.
\begin{exampleblock}{Example}
Input:\\
\texttt{``banana''}, position = 3, length = 4\
Result: \texttt{nana}
\end{exampleblock}    
\end{frame}

\begin{frame}[fragile]{Problem 3}{Solution}
\lstinputlisting{src/av9/p3.c}
\end{frame}

\begin{frame}{Problem 4}
Write a function that will check if one string is substring of some other
string.
\begin{exampleblock}{Example}
\texttt{``face''} is substring of \texttt{``Please faceAbook''}
\end{exampleblock}  
\end{frame}

\begin{frame}[fragile]{Problem 4}{Solution}
\lstinputlisting{src/av9/p4.c}
\end{frame}

\begin{frame}{Problem 5}
Write a function that will check if given string is palindrome. Palindrome is a
string that is read same from left to right and from right to left.
\begin{exampleblock}{Example palindrome words}
dovod\\
ana\\
kalabalak
\end{exampleblock}
\begin{scriptsize}
\emph{Homework:} Write a function that will check if given sentance is
palindrome. Ignore the empty spaces, interpunction characters and the case of
letters.
\end{scriptsize}
\begin{exampleblock}{Example sentance palindromes}
\begin{scriptsize}
Јadenje i pienje daj\\
A man, a plan, a canal, Panama\\
Never odd or even\\
Rise to vote sir
\end{scriptsize}
\end{exampleblock}

\end{frame}

\begin{frame}[fragile]{Problem 5}{Solution}
\lstinputlisting{src/av9/p5.c}
\end{frame}

\begin{frame}{Problem 6}
Write a function that for a given string will if it's complex enough to become a
password. Every password must have at least one letter, one digit and one
special character.
\begin{exampleblock}{Example}
\texttt{zdr@v0!} is valid password.\\
\texttt{zdravo} is not valid password.
\end{exampleblock}
\end{frame}

\begin{frame}[fragile]{Problem 6}{Solution}
\lstinputlisting{src/av9/p6.c}
\end{frame}

\begin{frame}{Problem 7}
Write a function that for will change the case of the letters and will remove
all digits and special characters.

\begin{exampleblock}{Example}
For the string: \texttt{``0v@ePr1m3R''} \\
The result should be: \texttt{``VEpRMr''} 
\end{exampleblock}
\end{frame}

\begin{frame}[fragile]{Problem 7}{Solution}
\lstinputlisting{src/av9/p7.c}
\end{frame}

\begin{frame}[fragile]{Problem 8}
Write a function that will trim a string (remove blanks at front and end of
string).
\begin{exampleblock}{Example}
For the string: \begin{verbatim}``   make trim   ''\end{verbatim} \\
The result should be: \texttt{``make trim''} 
\end{exampleblock}
\end{frame}

\begin{frame}[fragile]{Problem 8}{Solution}
\lstinputlisting{src/av9/p8.c}
\end{frame}
