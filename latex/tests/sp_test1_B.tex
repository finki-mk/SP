\documentclass[11pt]{examdesign}
\usepackage{ucs}
\usepackage[T2A]{fontenc}
\usepackage[utf8]{inputenc}
\usepackage{amsmath}
\usepackage{pifont}
\usepackage{verbatim}
\usepackage[ddmmyyyy]{datetime}
\renewcommand{\dateseparator}{.}
\SectionFont{\large\sffamily}
\usepackage[margin=1.5cm]{geometry}
%\Fullpages
\ContinuousNumbering
%\ShortKey
\DefineAnswerWrapper{}{}
\NumberOfVersions{1}
\IncludeFromFile{sp_test1_code.tex}

\def\namedata{Име и презиме: \hrulefill \\[5pt]
Индекс: \hrulefill}

\begin{examtop}
{\parbox{.5\textwidth}{\textbf{\classdata} \\
\examtype, 29.10.2014, Група: \fbox{\textsf{Б}}\\ \emph{Секое проашање има само \textbf{еден точен} одговор. Тестот нема негативни поени.}}
\hfill
\parbox{.45\textwidth}{\normalsize \namedata}}
\end{examtop}

\def\aftersectsep{0pt}
\def\beforesectsep{0pt}
\def\beforeinstsep{0pt}
\def\afterinstsep{0pt}


\class{\Large{Структурно програмирање}}
\examname{Тест 1}
\begin{document}
%\SectionPrefix{Дел \arabic{sectionindex}. \space}

\begin{multiplechoice}[title={},suppressprefix=yes,rearrange=no]

\begin{question}
Кој од следните изрази е \textbf{валиден} во C?
    \choice{\texttt{for(x = 1)}}
    \choice{\texttt{for(x > 0; x--)}}
    \choice[!]{\texttt{while(10 * 10)}}
    \choice{ниту еден}
\end{question}

\begin{question}
Кој е валиден код во C кој го проверува следниот услов 5 < x <
50?
    \choice{\texttt{while((50 >= x) \&\& (x > 5))}}
    \choice{\texttt{if((50 >= x) \&\& (x > 5))}}
    \choice[!]{\texttt{if((50 > x) \&\& (x > 5))}}
    \choice{\texttt{if(5 < x <= 50)}}
\end{question}

\begin{question}
Што ќе биде излезот од извршувањето на следниот програмски сегмент?
  \InsertChunk{c0}
  \choice{\texttt{0 1 2 3}}
  \choice{\texttt{2 1 0}}
  \choice[!]{\texttt{0 1 2}}
  \choice{the code is invalid}
\end{question}

\begin{question}
Кој од следните изрази не дава резултат 2.5 ако променливите се декларирани со \texttt{int a=5, b=2;}
  \choice{\texttt{a*1./b}}
  \choice{\texttt{float(a)/b}}
  \choice[!]{\texttt{(float)(a/b)}}
  \choice{\texttt{a/(float)b}}
\end{question}

\begin{question}
Што ќе биде излезот од извршувањето на следниот програмски сегмент?
    \InsertChunk{c4B}
    \choice{YES}
    \choice[!]{NO}
    \choice{не може да се предвиди, зависи од променливата \texttt{x}}
    \choice{кодот има синтаксна грешка}
\end{question}
  
\begin{question}
Што ќе биде излезот од извршувањето на следниот програмски сегмент?
    \InsertChunk{c1B}
    \choice{\texttt{b = 2.25}}
    \choice[!]{\texttt{b = 001}}  
    \choice{\texttt{b = 1}}               
    \choice{\texttt{b = 1.00}}
\end{question}

\begin{question}
Која ќе биде вредноста на \texttt{i} по извршување на на следниот програмски сегмент?
    \InsertChunk{c2B}
    \choice{\texttt{0}}
    \choice[!]{\texttt{11}}
    \choice{\texttt{9}}
    \choice{\texttt{10}}
\end{question}
  
\begin{question}
Што ќе биде излезот од извршувањето на следниот програмски сегмент?
    \InsertChunk{c5B}
    \choice[!]{\texttt{123}}
    \choice{\texttt{1}}
    \choice{\texttt{2}}
    \choice{\texttt{3}}
\end{question}
  
\begin{question}
Што ќе биде излезот од извршувањето на следниот програмски сегмент?
    \InsertChunk{c3}
    \choice[!]{0.0}
    \choice{2.0}
    \choice{2.5}
    \choice{3.0}
\end{question}
  
\begin{question}
Со кој од следните изрази се декларира низа \texttt{a} од 5 цели броеви?
    \choice[!]{\texttt{int a[] = \{1, 2, 3, 4, 5\};}}
    \choice{\texttt{int array[];}}
    \choice{\texttt{a int[5];}}
    \choice{\texttt{array int[5];}}
\end{question}
  
\end{multiplechoice}

\end{document}
