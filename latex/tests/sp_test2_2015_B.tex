\documentclass[11pt]{examdesign}
\usepackage{ucs}
\usepackage[T2A]{fontenc}
\usepackage[utf8]{inputenc}
\usepackage{amsmath}
\usepackage{pifont}
\usepackage{verbatim}
\usepackage[ddmmyyyy]{datetime}
\renewcommand{\dateseparator}{.}
\SectionFont{\large\sffamily}
\usepackage[margin=1.5cm]{geometry}
%\Fullpages
\ContinuousNumbering
%\ShortKey
\DefineAnswerWrapper{}{}
\NumberOfVersions{1}
\NoRearrange
\IncludeFromFile{sp_test2_code.tex}

\def\namedata{Име и презиме: \hrulefill \\[5pt]
Индекс: \hrulefill}

\begin{examtop}
{\parbox{.5\textwidth}{\textbf{\classdata} \\
\examtype, 14.12.2015, Група: \fbox{\textsf{Б}}\\ \emph{Секое проашање има само \textbf{еден точен} одговор.}}
\hfill
\parbox{.45\textwidth}{\normalsize \namedata}}
\end{examtop}

\def\aftersectsep{0pt}
\def\beforesectsep{0pt}
\def\beforeinstsep{0pt}
\def\afterinstsep{0pt}


\class{\Large{Структурно програмирање}}
\examname{Тест 2}
\begin{document}
%\SectionPrefix{Дел \arabic{sectionindex}. \space}


\begin{multiplechoice}[title={},suppressprefix=yes]

\begin{question}
Која од следните функции се користи за запишување знак во датотека во C?
    \choice[!]{\texttt{int fputc(int character, FILE *f);}}
    \choice{\texttt{void writef(file f)}}
    \choice{\texttt{void fileput(int c, F* file);}}
    \choice{\texttt{int putchar(int character);}}
\end{question}

\begin{question}
Што ќе биде излезот од извршувањето на следниот програмски сегмент?
\InsertChunk{c6B}
    \choice{\texttt{49 25 9 1 1 9 25 49}}
    \choice{49 25 9 1}
    \choice[!]{1 9 25 49}
    \choice{не може да се одреди}
\end{question}

\begin{question}
Која од следните декларации \textbf{НЕ} е валиден функциски прототип?
    \choice[!]{\texttt{double number;}}
    \choice{\texttt{int number(char, char);}}
    \choice{\texttt{void number();}}
    \choice{\texttt{float number();}}
\end{question}

\begin{question}

Што ќе биде излезот од извршувањето на следниот програмски сегмент?
\InsertChunk{c1B}
    \choice{ништо, грешка при компајлирање}
    \choice{\texttt{sloboda}}
    \choice{\texttt{vloboda}}
    \choice[!]{\texttt{boda}}
\end{question}

\begin{question}
Што ќе биде излезот од извршувањето на следниот програмски сегмент?
  \InsertChunk{c2B}
  \choice[!]{\texttt{5}}
  \choice{\texttt{1}}
  \choice{\texttt{3}}
  \choice{не може да се одреди}
\end{question}

\begin{question}
На кој начин се пристапува до последниот елемент во низата дефинирана со: 
\texttt{int broevi[] = \{ 5, 4, 3, 2, 1 \};}
    \choice{\texttt{*(broevi + 4 * sizeof(int))}}               
    \choice[!]{\texttt{*(broevi + 4)}}
    \choice{\texttt{broevi[5]}}  
    \choice{\texttt{broevi + 5}}
\end{question}

\begin{question}
Која ќе биде вредноста на \texttt{a} по извршувањето на следниот програмски сегмент?
    \InsertChunk{c3B}
    \choice[!]{\texttt{5}}
    \choice{\texttt{3}}
    \choice{\texttt{6}}
    \choice{недифинирана}
\end{question}

\begin{question}
Со која од следниве наредби се декларира дводимензионално поле (матрица) од цели броеви со димензија 7x7 (7 редици и 7 колони)?
\choice{\texttt{int a[7 x 7];}}
\choice{\texttt{int a[7, 7];}}
\choice{\texttt{int a[49];}}
\choice[!]{\texttt{int a[7][7];}}
\end{question}
  
\begin{question}
Даден е следниот програмски сегмент, кој е правилниот начин на повикување на функцијата дефинифрана со прототип \texttt{void find(char *а, int *b)}?
\InsertChunk{c11B}
    \choice[!]{\texttt{find(x, \&c);}}
    \choice{\texttt{find(\&x, \&c);}}
    \choice{\texttt{find(x, c);}}
    \choice{\texttt{find(\&x, c);}}
\end{question}
  
\begin{question}
Кој е правилниот израз во C за отварање на датотека \texttt{"zborovi.dat"} само за читање?
    \choice{\texttt{fopen("zborovi.txt","rb");}}
    \choice{\texttt{fopen("zborovi.dat","w");}}
    \choice[!]{\texttt{fopen("zborovi.dat", "r");}}
    \choice{\texttt{fopen("zborovi.dat", "read");}}
\end{question}

\end{multiplechoice}

\end{document}
